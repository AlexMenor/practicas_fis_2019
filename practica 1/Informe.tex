\documentclass{article}
\usepackage[utf8]{inputenc}
\usepackage[spanish]{babel}
\usepackage{graphicx}
\graphicspath{{./img/}}

\title{Práctica 1. Sistema de gestión para una agencia de viajes}
\author{Eugenio Alcántara García\\
		\and Noelia Escalera Mejías\\
		\and Alejandro Menor Molinero\\
		\and Jesús Torres Sánchez}
\begin{document}
	\maketitle
	\section{Descripción general del sistema y objetivos}
	\subsection{Descripción general del sistema}
	\subsection{Objetivos}
	\section{Lista inicial de requisitos}
	\subsection{Requisitos funcionales}
	\begin{enumerate}
		\item {\bf Realización de reservas online.} El usuario podrá realizar reservas a través de nuestra interfaz web.
		\item {\bf Realización de reservas presenciales.} Los empleados de la agencia podrán registrar reservas en el sistema si un cliente va a la agencia a solicitarla.
		\item {\bf Registro en página web.} El usuario podrá registrarse en nuestra página web y disfrutar de los beneficios que esto supone.
		\item {\bf Cambio de contraseña para usuarios registrados.} Todos los usuarios registrados en la página web pueden solicitar un cambio de contraseña.
		\item {\bf Personalización de perfil.} El usuario podrá personalizar su perfil, añadir sus datos personales y preferencias.
		\item {\bf Pago online.} El usuario podrá pagar de forma online sus reservas, viajes u otros servicios.
		\item {\bf Pago presencial.} Los empleados de la agencia podrán registrar los pagos que los clientes soliciten presencialmente.
		\item {\bf Pago a plazos.}
		\item {\bf Envío de publicidad al cliente.} El sistema podrá enviar publicidad a los usuarios tanto por correo como en la propia página.
		\item {\bf Consulta de circuitos turísticos creados por la agencia.}
		\item {\bf Gestión de circuitos turísticos.}
		\item {\bf El sistema debe permitir al cliente contratar un seguro de viaje.}
		\item {\bf El sistema debe enviar por correo una confirmación de cada pago realizado por el cliente.}
		\item {\bf Cupones de descuento y sistema de referidos.}
		\item {\bf Gestión de actividades turísticas.}
		\item {\bf Control de agentes comerciales.}
		\item {\bf Sistema de dar de alta.}
		\item {\bf El sistema debe poder realizar copias de seguridad de los datos.}
	\end{enumerate}
	\subsection{Requisitos no funcionales}
	\begin{enumerate}
		\item {\bf Publicidad personalizada.} La publicidad mostrada al usuario variará dependiendo de las preferencias de su perfil y de sus lugares favoritos.
		\item {\bf Confidencialidad de datos.} Los datos personales del usuario (datos de su perfil, historial de búsqueda, viajes, compras, etc) solo podrán ser accesibles por el propio usuario y el administrador/es de la agencia.
		\item {\bf Copia de seguridad periódica.} El sistema realizará una copia de seguridad cada 24h automáticamente.
		\item {\bf Copia de seguridad manual.} El administrador/es de la página web podrá realizar una copia de seguridad de los datos cuando desee.
	\end{enumerate}
	\subsection{Requisitos de información}
	\begin{enumerate}
		\item 
	\end{enumerate}
	\section{Glosario de términos}
\end{document}