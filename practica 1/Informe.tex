\documentclass{article}
\usepackage[utf8]{inputenc}
\usepackage[spanish]{babel}
\usepackage{graphicx}
\graphicspath{{./img/}}

\title{Práctica 1. Sistema de gestión para una agencia de viajes}
\author{Eugenio Alcántara García\\
		\and Noelia Escalera Mejías\\
		\and Alejandro Menor Molinero\\
		\and Jesús Torres Sánchez}
\begin{document}
	\maketitle
	\section{Descripción general del sistema y objetivos}
	\subsection{Descripción general del sistema}
	Se trata de un sistema que gestiona una agencia de viajes y que nos facilita todos los trámites relacionados con la misma (pagos, reservas, viajes, etc). Además permite que los clientes aporten sus propias valoraciones para así proporcionar un mejor servicio.
	\subsection{Objetivos}
	\begin{itemize}
		\item \textbf{OBJ-1.} Gestionar reservas y pagos de viajes.
		\begin{itemize}
			\item \textbf{OBJ-1.1.} Online.
			\item \textbf{OBJ-1.1.} En la agencia física.
		\end{itemize}
		\item \textbf{OBJ-2.} Gestionar las actividades turísticas.
		\item \textbf{OBJ-3.} Gestionar los agentes comerciales.	
	\end{itemize}
	\section{Lista inicial de requisitos}
	\subsection{Requisitos funcionales}
	\begin{itemize}
		\item {\bf RF-1. Realización de reservas.}
		\begin{itemize}
			\item \textbf{RF-1.1. Realización de reservas online.} El usuario podrá realizar reservas a través de nuestra interfaz web. 
			\item {\bf RF-1.1. Realización de reservas presenciales.} Los empleados de la agencia podrán registrar reservas en el sistema si un cliente va a la agencia a solicitarla.
		\end{itemize}
		\item \textbf{RF-2. Gestión de cuentas en la web.}
		\begin{itemize}
			\item {\bf RF-2.1. Registro en página web.} El usuario podrá registrarse en nuestra página web y disfrutar de los beneficios que esto supone.
			\item {\bf RF-2.2. Gestión y personalización del perfil.} Todos los usuarios registrados en la página web pueden solicitar un cambio de contraseña. Además, el usuario podrá personalizar su perfil, añadir sus datos personales.
		\end{itemize}
		\item \textbf{RF-3. Gestión de pagos.}
		\begin{itemize}
			\item {\bf RF-3.1. Pago online.} El usuario podrá pagar de forma online sus reservas, viajes u otros servicios.
			\item {\bf RF-3.2. Pago presencial.} Los empleados de la agencia podrán registrar los pagos que los clientes soliciten presencialmente.
			\item {\bf RF-3.3. Pago a plazos.} Tanto online como presencial.
		\end{itemize}
		
		\item \textbf{RF-4. Gestión de publicidad.}
		\begin{itemize}
			\item {\bf RF-4.1. Envío de publicidad al cliente.} El sistema podrá enviar publicidad a los usuarios tanto por correo como en la propia página.
			\item {\bf RF-4.2. Cupones de descuento y sistema de referidos.} En función de las opiniones que el cliente deje en la web y de los viajes que haga con la agencia.
			\item \textbf{RF-4.3. Gestión de preferencias.}
			\begin{itemize}
				\item \textbf{RF-4.3.1. Sistema de opinión.} Los usuarios podrán valorar su experiencia en alojamientos, destinos, actividades, circuitos y agentes de 1 a 5 estrellas y dejar una opinión.
				\item \textbf{RF-4.3.2. Evaluación de las preferencias del cliente.} Para enviar publicidad de acuerdo a sus gustos y mostrar recomendaciones.
			\end{itemize}
		\end{itemize}
	
		\item \textbf{RF-5. Gestión de circuitos y actividades turísticas.}
		\begin{itemize}
			\item \textbf{RF-5.1 Gestión de actividades}
			\begin{itemize}
				\item \textbf{RF-5.1.1. Contratar actividad turística.}
				\item \textbf{RF-5.1.2. Consultar actividades turísticas disponibles.}
				\item \textbf{RF-5.1.3. Contratar un seguro de viaje.}
			\end{itemize}
			\item \textbf{RF-5.2 Gestión de circuitos}
			\begin{itemize}
				\item \textbf{RF-5.2.1. Contratar un circuito.}
				\item \textbf{RF-5.2.2. Consulta de los circuitos creados por la agencia.}
				\item \textbf{RF-5.2.3. Crear un circuito.}
			\end{itemize}
		\end{itemize}
		
		\item \textbf{RF-6. Gestion de agentes comerciales.}
		\begin{itemize}
			\item \textbf{RF-6.1. Alta de un agente comercial.}
			\item \textbf{RF-6.2. Baja de un agente comercial.}
		\end{itemize}
		
		\item {\bf RF-7. Gestión de copias de seguridad.}
		\begin{itemize}
			\item \textbf{RF-7.1. Copia de seguridad manual.} Un empleado autorizado podrá realizar una copia de seguridad del sistema cuando crea conveniente.
			\item \textbf{RF-7.2. Copia de seguridad periódica.} El sistema realizará una copia de seguridad de los datos cada 24h automáticamente.
		\end{itemize}
	
		\item \textbf{RF-8. Gestión de viajes.}
		\begin{itemize}
			\item \textbf{RF-8.1. Selección del medio de transporte.}
			\item \textbf{RF-8.2. Selección de compañía.}
		\end{itemize}
	\end{itemize}
	\subsection{Requisitos no funcionales}
	\begin{itemize}
		\item {\bf RNF-1. Confidencialidad de datos.} Los datos personales del usuario (datos de su perfil, historial de búsqueda, viajes, compras, etc) solo podrán ser accesibles por el propio usuario y el/los administrador/es de la agencia. \textbf{Requisitos relacionados: RF-2, RF-6}.
		\item {\bf RNF-2. Confirmación de pago.} El sistema debe enviar por correo una confirmación de cada pago realizado por el cliente. \textbf{Requisitos relacionados: RF-3}.
		\item \textbf{RNF-3. Interfaz sencilla e intuitiva.} Los trámites se podrán realizar en pocos pasos. \textbf{Requisitos relacionados: RF-1, RF-2, RF-3, RF-5}.
		\item \textbf{RNF-4. Disponibilidad del sistema.} El sistema debe estar operativo siempre y en caso de fallo debe recuperarse en menos de 2 horas.
		\item \textbf{RNF-5. Tiempo de espera.} El usuario no debe esperar más de 1 minuto para que se realicen sus pagos. \textbf{Requisitos relacionados: RF-3}.
		\item \textbf{RNF-5. Accesibilidad sitio web.} El sitio web será accesible desde todo tipo de dispositivos con conexión a internet. \textbf{Requisitos relacionados: RF-2}.
		
	\end{itemize}
	\subsection{Requisitos de información}
	\begin{itemize}
		\item \textbf{RI-1. Destinos.} Fotos, descripción breve y sitios de interés. \textbf{Requisitos relacionados: RF-5}.
		\item \textbf{RI-2. Hoteles y hostales.} Localización, fotos y categoría. \textbf{Requisitos relacionados: RF-5.}
		\item \textbf{RI-3. Actividades y circuitos.} Descripción, fotos y horarios. \textbf{Requisitos relacionados: RF-5}.
		\item \textbf{RI-4. Datos de agentes comerciales.} Datos personales, detalles del contrato y currículum. \textbf{Requisitos relacionados: RF-6}.
		\item \textbf{RI-5. Datos de clientes.}
		\begin{itemize}
			\item \textbf{RI-5.1. Datos personales.}
			\item \textbf{RI-5.2. Datos de reservas y pagos.}
			\item \textbf{RI-5.3. Datos de preferencias.}
		\end{itemize}
		\textbf{Requisitos relacionados: RF-1, RF-2, RF-3, RF-4, RF-5}.
	\end{itemize}
	\section{Glosario de términos}
	\begin{itemize}
		\item \textbf{Actividad turística.} Actividad que se realiza en un determinado destino (visita de museos, excursión, etc).
		\item \textbf{Circuito turístico.} Conjunto de actividades a realizar en un destino turístico.
		\item \textbf{Agente comercial.} Empleado de la agencia que trabaja en una de sus oficinas físicas.
		\item \textbf{Cupón de descuento.} Código canjeable en la página web que proporciona descuentos para viajes y actividades.
		\item \textbf{Empleado autorizado.} Empleado que tiene permiso para realizar copias de seguridad del sistema, así como otras gestiones relacionadas con éste.
		\item \textbf{Administrativo.} Empleado encargado de realizar los trámites especializados del sistema.
	\end{itemize}
\end{document}